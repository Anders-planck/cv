% article class because we want to fully customize the page and not use a cv template
\documentclass[a4paper,12pt]{article}

%----------------------------------------------------------------------------------------
%	FONT
%----------------------------------------------------------------------------------------

% % fontspec allows you to use TTF/OTF fonts directly
% \usepackage{fontspec}
% \defaultfontfeatures{Ligatures=TeX}

% % modified for ShareLaTeX use
% \setmainfont[
% SmallCapsFont = Fontin-SmallCaps.otf,
% BoldFont = Fontin-Bold.otf,
% ItalicFont = Fontin-Italic.otf
% ]
% {Fontin.otf}

%----------------------------------------------------------------------------------------
%	PACKAGES
%----------------------------------------------------------------------------------------
\usepackage{url}
\usepackage{parskip} 	

%other packages for formatting
\RequirePackage{color}
\RequirePackage{graphicx}
\usepackage[usenames,dvipsnames]{xcolor}
\usepackage[scale=0.9]{geometry}

%tabularx environment
\usepackage{array}
\usepackage{tabularx}

%for lists within experience section
\usepackage{enumitem}

% centered version of 'X' col. type
\newcolumntype{C}{>{\centering\arraybackslash}X} 

%to prevent spillover of tabular into next pages
\usepackage{supertabular}
\newlength{\fullcollw}
\setlength{\fullcollw}{0.47\textwidth}

%custom \section
\usepackage{titlesec}				
\usepackage{multicol}
\usepackage{multirow}

%CV Sections inspired by: 
%http://stefano.italians.nl/archives/26
\titleformat{\section}{\Large\scshape\raggedright}{}{0em}{}[\titlerule]
\titlespacing{\section}{0pt}{10pt}{10pt}

%for publications
\usepackage[style=authoryear,sorting=ynt, maxbibnames=2]{biblatex}

%Setup hyperref package, and colours for links
\usepackage[unicode, draft=false]{hyperref}
\definecolor{linkcolour}{rgb}{0,0.2,0.6}
\hypersetup{colorlinks,breaklinks,urlcolor=linkcolour,linkcolor=linkcolour}
\addbibresource{citations.bib}
\setlength\bibitemsep{1em}

%for rounded images
\usepackage{tikz}
\newcommand{\roundpic}[1]{\begin{tikzpicture}\clip (0,0) circle (2cm);\node at (0,0) {\includegraphics[width=4cm]{#1}};\end{tikzpicture}}

%debug page outer frames
%\usepackage{showframe}


% job listing environments
\newenvironment{jobshort}[2]
    {
    \begin{tabularx}{\linewidth}{@{}l X r@{}}
    \textbf{#1} & \hfill &  #2 \\[3.75pt]
    \end{tabularx}
    }
    {
    }

\newenvironment{joblong}[2]
    {
    \begin{tabularx}{\linewidth}{@{}l X r@{}}
    \textbf{#1} & \hfill &  #2 \\[3.75pt]
    \end{tabularx}
    \begin{minipage}[t]{\linewidth}
    \begin{itemize}[nosep,after=\strut, leftmargin=1em, itemsep=3pt,label=--]
    }
    {
    \end{itemize}
    \end{minipage}    
    }



%----------------------------------------------------------------------------------------
%	BEGIN DOCUMENT
%----------------------------------------------------------------------------------------
\begin{document}

% non-numbered pages
\pagestyle{empty} 

%----------------------------------------------------------------------------------------
%	TITLE
%----------------------------------------------------------------------------------------

% \begin{tabularx}{\linewidth}{ @{}X X@{} }
% \huge{Your Name}\vspace{2pt} & \hfill \emoji{incoming-envelope} email@email.com \\
% \raisebox{-0.05\height}\faGithub\ username \ | \
% \raisebox{-0.00\height}\faLinkedin\ username \ | \ \raisebox{-0.05\height}\faGlobe \ mysite.com  & \hfill \emoji{calling} number
% \end{tabularx}

\begin{center}
    \roundpic{./resources/me.jpg}
\end{center}

\begin{tabularx}{\linewidth}{@{} C @{}}
\Huge{ANDERS Planck} \\[7.5pt]
\href{https://github.com/Anders-planck}{\raisebox{-0.05\height}\faGithub\ Anders-planck} \ $|$ \
\href{https://linkedin.com/in/anders-planck}{\raisebox{-0.05\height}\faLinkedin\ anders-planck} \ $|$ \
\href{mailto:anders.jipwouo@gmail.com}{\raisebox{-0.05\height}\faEnvelope \ anders.jipwouo@gmail.com} \ $|$ \
\href{tel:+393518884983}{\raisebox{-0.05\height}\faMobile \ +39 3518884983} \\
\normalsize{Ferrara, Emilia-Romagna}
\end{tabularx}

%----------------------------------------------------------------------------------------
% EXPERIENCE SECTIONS
%----------------------------------------------------------------------------------------

%Interests/ Keywords/ Summary
\section{Chi sono}
Sviluppatore Full Stack con circa 3 anni di esperienza, specializzato in PHP e tecnologie web moderne, con una forte passione per il design di soluzioni scalabili e orientate a best practice. Ho contribuito attivamente a progetti open source e privati, occupandomi di backend, frontend, API, infrastruttura e interfacce utente.
Sono abituato a lavorare in team adottando metodologie Agile / Scrum, e a mantenere un forte orientamento alla qualità, testing e manutenibilità del codice.

%Experience
\section{A-Cube S.R.L. - Sviluppatore Full Stack}
{Gennaio 2023 - in corso}

\begin{joblong}{\href{https://github.com/a-cube-io/ereceipts-js-sdk}{A-Cube e-Receipts SDK, (Pacchetto NPM)}}{Luglio 2025 - in corso}
\item SDK per l'integrazione con il sistema di gestione scontrini elettronici A-Cube
\item Sviluppatore Full Stack TypeScript
\item SDK altamente performante e modulare per l'integrazione con le API RESTful di ACube per la gestione degli scontrini elettronici
\item Progettazione e sviluppo dell'architettura modulare dell'SDK con supporto per ambienti online e offline
\item Implementazione di un sistema di autenticazione robusto con supporto mTLS e gestione della cache
\item Sviluppo di un sistema di coda per le operazioni offline con sincronizzazione automatica
\item Implementazione di un sistema di cache intelligente per ottimizzare le prestazioni
\item Sviluppo di API client tipizzate per l'interazione con i servizi ACube (gestione merchant, scontrini, transazioni, reportistica)
\item Scrittura di test automatici per garantire l'affidabilità e documentazione dettagliata
\end{joblong}

\begin{joblong}{\href{https://www.npmjs.com/package/@a-cube-io/expo-mutual-tls}{Modulo Expo: @a-cube-io/expo-mutual-tls}}{Luglio 2025 - in corso}
\item Modulo Expo open-source per l'implementazione di autenticazione client Mutual TLS (mTLS) in applicazioni React Native/Expo
\item Sviluppatore Full Stack
\item Progettazione e sviluppo del modulo Expo in TypeScript con implementazioni native per iOS (Swift) e Android (Kotlin)
\item Definizione e implementazione di API semplici per configurazione, gestione certificati e operazioni di rete
\item Sviluppo di funzionalità avanzate come supporto biometrico e gestione eventi
\item Gestione di integrazione con servizi esterni e pubblicazione su npm
\item Sviluppo di esempi completi e documentazione esaustiva
\end{joblong}

\begin{joblong}{Applicazioni Web/Mobile, (Dashboard Pem-Pel)}{Guigno 2025 - in corso}
\item Sviluppatore Full Stack (React Native/TypeScript)
\item Sviluppo e manutenzione di un'applicazione mobile e web cross-platform per la gestione e il monitoraggio delle attività commerciali
\item Sviluppo di interfacce utente reattive e performanti con React Native ed Expo
\item Implementazione di componenti riutilizzabili con TypeScript per garantire la type-safety
\item Gestione dello stato dell'applicazione tramite React Context e React Query
\item Implementazione di animazioni fluide e transizioni tra schermate
\item Ottimizzazione delle prestazioni con tecniche di memoizzazione e virtualizzazione delle liste
\item Progettazione di un'architettura modulare e scalabile con React Navigation
\end{joblong}

\begin{joblong}{DASHBORD: Fatture/PA e Tutti A-Cube Product}{Gennaio 2025 - in corso}
\item Sviluppatore Frontend \& UI/UX Designer
\item Rivisitazione completa di un'applicazione web per la gestione e l'elaborazione di fatture elettroniche per la Pubblica Amministrazione
\item Tecnologie principali: React 18 con TypeScript, Redux Toolkit, Material-UI (MUI) con tema personalizzato
\item Migrazione da classi a Functional Components con React Hooks e architettura a feature
\item Riduzione del bundle size del 40\% con code-splitting e ottimizzazione delle chiamate API
\item Interfaccia completamente riprogettata con Material-UI v5 e gestione avanzata degli stati
\item Supporto completo FatturaPA 1.2.1 e integrazione con il sistema di conservazione a norma
\item \textbf{Risultati}: +65\% di performance complessive, 4.8/5 valutazione media utente
\end{joblong}

\begin{joblong}{Open Banking UI}{Gennaio 2024 - Settembre 2024}
\item Sviluppatore Full Stack
\item Applicazione web innovativa per l'open banking per facilitare nuovi metodi di pagamento tramite bonifici SEPA e istantanei
\item Sviluppo architetturale completo con SPA React 18 con TypeScript utilizzando Vite come build tool
\item Sistema di pagamenti avanzato per bonifici SEPA e SEPA Instant con gestione completa degli stati
\item Sviluppo di un sistema flessibile per pagamenti rateali con selezione multipla e validazione real-time
\item API RESTful robuste con integrazione tramite Redux Toolkit Query e validazione end-to-end
\item Interfaccia utente moderna con design system Material-UI (MUI) e tema personalizzato
\item Internazionalizzazione completa con React Intl, italiano come lingua principale
\item Integrazione di PDF.js per la preview e gestione di fatture e documenti di pagamento
\item Ambiente multi-stage con configurazione e deployment su tre ambienti
\end{joblong}

\begin{joblong}{Integrazioni e Dashboard (Stripe Dashboard)}{Gennaio 2024 - Decembre 2024}
\item Sviluppatore Full Stack
\item Sviluppo di una dashboard avanzata per la gestione e il monitoraggio di documenti Stripe
\item Progettazione e sviluppo del backend con gestione di API REST per l'interazione sicura tra frontend e servizi Stripe
\item Sviluppo frontend con React / TypeScript / UI responsive utilizzando componenti modulari
\item Gestione database e dati strutturati tramite servizi di query ottimizzate
\item Integrazione con servizi esterni come Stripe per autenticazione e autorizzazioni
\item Implementazione di test automatici per componenti e flussi critici
\item Collaborazione in team con metodologia Agile / Scrum
\end{joblong}

\begin{joblong}{A-Cube E-Invoicing - Stripe App UI Extension}{Gennaio 2024 -  in corso}
\item Sviluppatore Full Stack
\item Sviluppo di un'estensione enterprise-grade per la dashboard Stripe per la gestione completa di documenti di pagamento
\item Architettura context-based a 6 livelli utilizzando React Context + useReducer senza Redux
\item Sviluppo di 8 viste principali registrate in Stripe con pattern BaseView enterprise-grade
\item Implementazione avanzata di validazione per codice fiscale, partita IVA, codice destinatario B2B
\item Form Architecture con wizard step-by-step utilizzando react-hook-form + Zod validation
\item Ottimizzazione con 173 istanze di memoization e gestione intelligente del re-rendering
\item Integrazione sicura con API Stripe e sistemi contabili A-Cube
\end{joblong}

\begin{joblong}{PRESERVATION API}{Maggio 2023 -  Decembre 2024}
\item Sviluppatore Full Stack
\item Sviluppo e manutenzione di un servizio avanzato di conservazione elettronica digitale conforme alle normative italiane
\item Tecnologie e Architettura: Symfony 6.2+ con API Platform 3, Doctrine ORM con PostgreSQL, JWT con Lexik
\item Cloud: AWS SDK per integrazione S3 e Lambda, testing con PHPUnit, Behat, PHPStan livello 8
\item Containerizzazione: Docker con docker-compose per sviluppo locale e CI/CD
\item Progettazione API RESTful con API Platform 3, implementazione di 15+ risorse specifiche
\item Sviluppo architettura a microservizi con pattern CQRS per separazione responsabilità
\item Implementazione metadati strutturati per ogni tipologia documentale secondo normative AgID
\item Integrazione AWS per storage sicuro e processamento asincrono dei pacchetti
\item Test automation avanzata con $>$200 scenari Behat per test end-to-end
\end{joblong}

\begin{joblong}{Sistema di Notifiche Email Aziendale (Notifier)}{Maggio 2023 - Settembre 2023}
\item Sviluppatore Full Stack
\item Progetto enterprise per la gestione centralizzata delle notifiche email in ambiente AWS serverless
\item Architettura e sviluppo del servizio notifiche con funzione Lambda in Python 3.12
\item Sistema di generazione template email basato su HTML/Tailwind CSS con generazione automatizzata
\item Componente di tracciamento per monitorare eventi SES e salvare dati in DynamoDB
\item Integrazione API Platform con API Gateway e autorizzazioni granulari
\item Testing completo con copertura $>$80\% utilizzando pytest e unittest.mock
\item CI/CD e DevOps con AWS SAM e Docker Compose per sviluppo locale
\item Gestione infrastrutture con stack CloudFormation completo
\end{joblong}

%----------------------------------------------------------------------------------------
%	PROJECTS
%----------------------------------------------------------------------------------------
\section{Progetti Personali}

\begin{joblong}{VueKit}{2022}
\item Sviluppatore Full Stack
\item Kit di componenti e applicazione SaaS con Next.js e Firebase per gestione contenuti digitali (blog, tutorial, lavori) con autenticazione, database real-time e componenti UI modulari.
\item Sviluppo di oltre 65 componenti React utilizzando Chakra UI, TipTap per editor avanzati, Framer Motion per animazioni e integrazione con Firebase Functions/Firestore.
\item Gestione ruoli utente, ottimizzazioni performance (Tailwind CSS, Chart.js, Vercel deployment) e debugging per sincronizzazione dati e UX fluida.
\end{joblong}

%----------------------------------------------------------------------------------------
%	EDUCATION
%----------------------------------------------------------------------------------------
\section{Formazione}
\begin{tabularx}{\linewidth}{@{}l X@{}}
Settembre 2023 - in corso ( Fine prevista per Decembre 2025) & INGEGNERIA INFORMATICA E DELL’AUTOMAZIONE \textbf{Università degli Studi di Ferrara} \\
Settembre 2020 - Guigno 2023 & Laurea in Informatica ed Elettronica  \textbf{Università degli Studi di Ferrara} \\
\end{tabularx}

%----------------------------------------------------------------------------------------
%	LANGUAGES
%----------------------------------------------------------------------------------------
\section{Conoscenze Linguistiche}
\begin{tabularx}{\linewidth}{@{}l X@{}}
Francese & Madrelingua \\
Italiano & B2 \\
Inglese & B1 \\
\end{tabularx}


%----------------------------------------------------------------------------------------
%	SKILLS
%----------------------------------------------------------------------------------------
\section{Competenze Tecniche}

\begin{tabularx}{\linewidth}{|>{\raggedright\arraybackslash}p{4cm}|X|}
\hline
\textbf{Backend / Server} & PHP (Laravel, Symfony), Node.js (Adonis.js) \\
\hline
\textbf{Frontend / UI} & JavaScript, TypeScript, React, React Native, CSS3, HTML5, Responsive Design \\
\hline
\textbf{API / Comunicazione} & RESTful API, JSON, WebSocket \\
\hline
\textbf{Database} & MySQL, Relazionali e NoSQL \\
\hline
\textbf{DevOps / Cloud / Infrastruttura} & Linux, Docker, CI/CD, Firebase/Supabase (progetti personali)  \\
\hline
\textbf{Metodologie / Qualità} & Agile, Scrum, Test automatici (unit, integrazione), Git, Code Review \\
\hline
\textbf{Altre competenze} & Librerie open source, Contribuzione a progetti pubblici, Ottimizzazione prestazioni \\
\hline
\end{tabularx}

%----------------------------------------------------------------------------------------
%	MOTIVATION
%----------------------------------------------------------------------------------------
\section{Motivazione \& Obiettivo Professionale}

Desidero entrare in un contesto stimolante come Full Stack Developer PHP in cui poter contribuire attivamente al ciclo completo di sviluppo (da analisi, progettazione, sviluppo a deployment), crescere tecnicamente e collaborare con team strutturati su progetti enterprise. Il mio obiettivo è apportare valore fin da subito grazie alle mie competenze tecniche, senso del dovere e volontà di apprendere ulteriormente.

%----------------------------------------------------------------------------------------
%	ADDITIONAL INFO
%----------------------------------------------------------------------------------------
\section{Soft Skills}
\begin{tabularx}{\linewidth}{@{}l X@{}}
Problem Solving & Lavoro in Team \\
Adattabilità & Orientamento alla Qualità \\
\end{tabularx}

\vfill
\center{\footnotesize Last updated: \today}

\end{document}
